\documentclass{uulm-exam}

\usepackage{algorithmic}
\usepackage{tabularx}
\usepackage{pdfpages}
\usepackage{subfigure}
\usepackage{hyphenat}
\usepackage{wrapfig}
\usepackage[weather]{ifsym}
\usepackage{wasysym}
\usepackage[font=bf,justification=raggedright, singlelinecheck=false]{caption}

\lstset{language=JAVA,numbers=none}

\setboolean{showsolutions}{false}
% \setboolean{showsolutions}{true}

\usepackage{cmbright}

% Feld für Codewort für den Aushang zeigen
\setboolean{showcodeword}{false}
\setboolean{showbonus}{true}

\newcommand{\gt}{$>$}
\newcommand{\lt}{$<$}

\begin{document}
\title{Programmierung von Systemen}
\date{25.09.2012}
\institute{Institut für Verteilte Systeme}
\duration{90 Minuten}
\examiner{Prof. Franz J. Hauck}

\begin{hints}
\begin{itemize}
\item Bitte prüfen Sie die Vollständigkeit Ihrer Aufgabenblätter
(insgesamt \gettotalassignments{} Aufgaben auf  \ref{TotPages} Seiten)!
\item Lösungen bitte nur auf die Aufgabenblätter und nicht mit Rot- oder
Bleistift schreiben!
\item Als Schmierzettel können die Rückseiten verwendet werden.
Sollten sich Teile Ihrer Lösung nicht direkt bei der Aufgabe befinden, so
kennzeichnen Sie dies bitte deutlich bei der jeweiligen Aufgabe!
\end{itemize}
\end{hints}

\maketitle

%setzen des grafikpfades
\graphicspath{{aufgaben/img/}}

% einbinden der einzelnen Aufgaben und Lösungen
\begin{assignments}
\assignment{Datenbanksysteme}

\begin{enumerate}

\item 
Erläutern Sie den Begriff des Lost-Updates (verlorengegangene Änderung) im
Zusammenhang mit Transaktionen an einem Beispiel.

\begin{solution}{5}
Wenn zwei Transaktionen je den selben Wert eines Datums
lesen und unabhängig voneinander verändern und daraufhin wieder abspeichern,
geht die Aktualisierung der ersten speichernden Transaktion verloren. Beispiel:
T1 erhöht Wert um 1, T2 verdoppelt den Wert. Beide lesen Wert 4. T1 speichert
zuerst 5, T2 speichert danach 8 zurück. Die Aktualisierung auf 5 geht verloren.
\end{solution}

\subpoints{3}

\item 
Was versteht man unter dem Serialisierbarkeitsprinzip von Transaktionen? 

\begin{solution}{6}[false]
Eine nebenläufige Ausführung von Transaktionen ist genau dann korrekt, wenn es
eine serielle Ausführung dieser Transaktionen gibt, die vom selben
Ausgangszustand zum selben Endzustand kommt.
\end{solution}

\subpoints{5}

\item
Warum ist es sinnvoll beim Datenbankentwurf eine Normalform einzuhalten?

\begin{solution}{4}
Ein Entwurf gemäß Normalformen vermeidet Einfüge- und Löschanomalien, d.h. dass
unvollständige Daten nicht eingefügt oder wichtige Daten zusammen mit zu
löschenden Daten mitgelöscht werden müssen.
\end{solution}

\subpoints{2}

\end{enumerate}

\end{assignments}

% Zusatzblätter
%\identifier{PVS 2012}
%\makeemptysheet

% JavaDoc
%\includepdf,angle=90,nup=1x2]{API-Aufgabe-3}
% \includepdf[pages={4,1,3,2},angle=90,nup=1x2]{coffee-javadoc}

\end{document}