%%%%%%%%%%%%%%%%%%%%%%%%%%%%%%%%%%%%%%%%%%%%%%%%%%%%%%%%%%%%%%%%%%%%%%%%%
%                          Package Configs                              %
%%%%%%%%%%%%%%%%%%%%%%%%%%%%%%%%%%%%%%%%%%%%%%%%%%%%%%%%%%%%%%%%%%%%%%%%%

\microtypesetup{protrusion=true,expansion=true,tracking=true}

\hypersetup{hidelinks,colorlinks=true,linkcolor=uulm-in-link,urlcolor=uulm-in-link}

%%%%%%%%%%%%%%%%%%%%%%%%%%%%%%%%%%%%%%%%%%%%%%%%%%%%%%%%%%%%%%%%%%%%%%%%%
%                          Additional Colors                            %
%%%%%%%%%%%%%%%%%%%%%%%%%%%%%%%%%%%%%%%%%%%%%%%%%%%%%%%%%%%%%%%%%%%%%%%%%

\definecolor{red}{HTML}{a51d2d}
\definecolor{green}{HTML}{26a269}
\definecolor{blue}{HTML}{1a5fb4}
\definecolor{yellow}{HTML}{f5c211}
\definecolor{gray}{HTML}{c0bfbc}
\definecolor{darkgray}{HTML}{77767b}
\definecolor{lightgray}{HTML}{deddda}

%%%%%%%%%%%%%%%%%%%%%%%%%%%%%%%%%%%%%%%%%%%%%%%%%%%%%%%%%%%%%%%%%%%%%%%%%
%                          Common Lengths                               %
%%%%%%%%%%%%%%%%%%%%%%%%%%%%%%%%%%%%%%%%%%%%%%%%%%%%%%%%%%%%%%%%%%%%%%%%%

\newlength\@linewidth\@linewidth1px
\newlength\@rounding\@rounding1mm

%%%%%%%%%%%%%%%%%%%%%%%%%%%%%%%%%%%%%%%%%%%%%%%%%%%%%%%%%%%%%%%%%%%%%%%%%
%                               TikZ                                    %
%%%%%%%%%%%%%%%%%%%%%%%%%%%%%%%%%%%%%%%%%%%%%%%%%%%%%%%%%%%%%%%%%%%%%%%%%

\usetikzlibrary{shapes,arrows,arrows.meta,positioning,backgrounds}
\tikzset{
	lw/.style={line width=\@linewidth},
	lcr/.style={line cap=round},
	rnd/.style={rounded corners=\@rounding},
	o/.style={remember picture,overlay},
	squarenode/.style={rectangle, fill=white, draw=black, lw},
	roundednode/.style={squarenode, rnd},
	roundnode/.style={squarenode, circle},
	pill/.style={squarenode, rounded rectangle},
	arrow/.style={black,lw,-{Latex[round]},lcr,rnd},
	textarrow/.style={arrow,darkgray,short=1mm,->,-{Kite[#1]}},
	textarrow/.default=open,
	short/.style n args={1}{shorten <=#1,shorten >=#1},
	pipelinestep/.style={lw,rnd,shape=signal,signal from=west,signal pointer angle=130,minimum width=3cm,minimum height=17mm,draw=black,fill=uulm-akzent!10!white},
}

\pgfdeclarelayer{foreground}
\pgfsetlayers{background,main,foreground}

\tikzset{%
	on foreground layer/.style={%
			execute at begin scope={%
					\pgfonlayer{foreground}%
					\let\tikz@options=\pgfutil@empty%
					\tikzset{every on foreground layer/.try,#1}%
					\tikz@options%
				},
			execute at end scope={\endpgfonlayer}
		},
	on layer/.code={%
			\pgfonlayer{#1}\begingroup
			\aftergroup\endpgfonlayer
			\aftergroup\endgroup
		},
	node on layer/.code={%
			\gdef\node@@on@layer{%
				\setbox\tikz@tempbox=\hbox\bgroup\pgfonlayer{#1}\unhbox\tikz@tempbox\endpgfonlayer\egroup}
			\aftergroup\node@on@layer
		},
}

\def\node@on@layer{\aftergroup\node@@on@layer}

%%%%%%%%%%%%%%%%%%%%%%%%%%%%%%%%%%%%%%%%%%%%%%%%%%%%%%%%%%%%%%%%%%%%%%%%%
%                              Listings                                 %
%%%%%%%%%%%%%%%%%%%%%%%%%%%%%%%%%%%%%%%%%%%%%%%%%%%%%%%%%%%%%%%%%%%%%%%%%

\RequirePackage{listings}
\RequirePackage{accsupp}
\RequirePackage{lstautogobble}

\lstdefinestyle{default}{
	basicstyle=\fontsize{\f@size}{\f@baselineskip}\ttfamily\selectfont,
	keywordstyle=\bfseries\color{@lst@keyword},
	commentstyle=\itshape\color{darkgray},
	stringstyle=\itshape\color{darkgray},
	numbers=left,
	breaklines=false,
	keepspaces=true,
	captionpos=b,
	inputencoding=utf8,
	literate={~}{\ttfamily$\sim$}{1} {ä}{{\"a}}1 {ö}{{\"o}}1 {ü}{{\"u}}1 {Ä}{{\"A}}1 {Ö}{{\"O}}1 {Ü}{{\"U}}1,
	autogobble=true,
	tabsize=2,
	escapeinside={(*}{*)},
	showtabs=true,
	tab={\color{lightgray}\@lst@nocopy{\guilsinglright}\hfill},
	numberstyle=\color{gray!70}\scriptsize\@lst@nocopy,
	backgroundcolor=\color{@lst@background},
	rulecolor=\color{@lst@rule},
	framerule=\@linewidth,
	frame=single,
	prebreak=\rlap{\@lst@nocopy{$\hookleftarrow$}},
	fontadjust=true,
}
\lstdefinestyle{inl}{
	style=default,
	keywordstyle=\color{@lst@keyword},
	columns=fullflexible,
}
\lstdefinestyle{plain}{
	style=default,
	language=,
	backgroundcolor=\color{white},
	numbers=none,
	frame=none,
	tab={},
}

\def\@lst@nocopy#1{\BeginAccSupp{ActualText={}}#1\EndAccSupp{}}
\if@nocolor
	\colorlet{@lst@background}{white}
	\colorlet{@lst@rule}{black}
	\colorlet{@lst@keyword}{black}
\else
	\colorlet{@lst@background}{uulm-akzent!5!white}
	\colorlet{@lst@rule}{@lst@background!70!black}
	\colorlet{@lst@keyword}{uulm-in-link}
\fi
\lstset{style=default}

\def\inlineblock#1{\tikz[anchor=base,baseline]\node[inner sep=0.1818em,rnd,fill=gray!15,anchor=base,baseline=] {#1};}
\newcommand\code[2][]{\inlineblock{\lstinline[style=inl,language=#1]|#2|}}
\newcommand\plaincode[2][]{\lstinline[style=inl,language=#1]|#2|}
\def\codename#1{\textsc{#1}}
\def\filename#1{\textsl{#1}}
\def\lstoutput#1{\color{gray!80!black}#1}


%%%%%%%%%%%%%%%%%%%%%%%%%%%%%%%%%%%%%%%%%%%%%%%%%%%%%%%%%%%%%%%%%%%%%%%%%
%                              Lists                                    %
%%%%%%%%%%%%%%%%%%%%%%%%%%%%%%%%%%%%%%%%%%%%%%%%%%%%%%%%%%%%%%%%%%%%%%%%%

\setlist[itemize,1]{label=$\bullet$}
\setlist[description]{align=right,labelindent=-1.2ex,font=\normalfont\itshape,labelsep=\dimexpr-\itemindent+1.2ex\relax,leftmargin=0cm}

\newlist{inlenum}{enumerate*}{1}
\setlist[inlenum]{label=\emph{\roman*)},itemjoin={{, }},itemjoin*={{ und }}}

%%%%%%%%%%%%%%%%%%%%%%%%%%%%%%%%%%%%%%%%%%%%%%%%%%%%%%%%%%%%%%%%%%%%%%%%%
%                            Other Macros                               %
%%%%%%%%%%%%%%%%%%%%%%%%%%%%%%%%%%%%%%%%%%%%%%%%%%%%%%%%%%%%%%%%%%%%%%%%%

% \link <display> to <url>;
\def\link #1 to #2;{%
	\def\reduline{\bgroup\markoverwith{\textcolor{uulm-in-link}{\rule[-0.2ex]{1pt}{0.08pt}}}\ULon}%
	\reduline{\href{#2}{#1}}%
}

%%%%%%%%%%%%%%%%%%%%%%%%%%%%%%%%%%%%%%%%%%%%%%%%%%%%%%%%%%%%%%%%%%%%%%%%%
%                           Fancy Underlines                            %
%%%%%%%%%%%%%%%%%%%%%%%%%%%%%%%%%%%%%%%%%%%%%%%%%%%%%%%%%%%%%%%%%%%%%%%%%

\usetikzlibrary{tikzmark}
\def\m#1{\tikzmark{#1}}

\tikzset{ul/.style={#1,opacity=0.35,rounded corners=.17ex}}
% \ul[<y shift>]{<color>}{<start>}{<end>}{<underline name>}
% <start> and <end> are names created with \m{<name>}
% <underline name> can later be referenced in tikzpictures and will point to the
% center of the underline
% There is also <underline name>_west and <underline name>_east which point to
% the --- you guessed it --- eastern and western end of the underline
%
% This is \m{s}a nice\m{e} example
% \ul{red}{s}{e}{u}
% \begin{tikzpicture}[o] % o is relevant for "u" to be accessible
%     \node at (u) {Hi!};
% \end{tikzpicture}
\newcommand\ul[5][0pt]{\begin{tikzpicture}[o]
	\coordinate(#5_west) at ([yshift=#1-0.2ex]pic cs:#3);
	\coordinate(#5_east) at ([yshift=#1+0.4ex]pic cs:#4);
	\fill[ul=#2] (#5_west) rectangle node(#5){} (#5_east);
\end{tikzpicture}}