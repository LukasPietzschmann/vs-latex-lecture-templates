% Jede Hauptaufgabe hat eine Überschrift und die Gesamtpunkte
\assignment{Fenstersysteme und Präsentation}

% Teilaufgabe als enumerate items
\begin{enumerate}

\item Bei einer Smartphone-Benutzeroberfläche kann immer nur eine Applikation
gleichzeitig angezeigt werden. Welche Komponente eines Fenstersystems --
verglichen mit denen eines normalen PC-Systems -- kann dadurch erheblich vereinfacht
werden? Begründen Sie Ihre Antwort.
\subpoints{2} % Die Punkte für die Teilaufgabe

% Zu jeder Teilaufgabe ein solution-Block der form \begin{solution}{Platz für Lösung}[Linien true/false] ... \end{solution}
\begin{solution}{5}[squared]
Fenstermanager. Fenster können sich nicht mehr überlappen,
Fokus/Eventzustellung/damage/redraw entfällt 
\end{solution}

\item Mit welcher Methode kann die Bildtiefe und damit der Speicherbedarf des
Framebuffers reduziert werden, ohne die Anzahl an darstellbaren Farben
einzuschränken? Skizzieren Sie die dafür nötigen Speicherstrukturen. Bei welcher
Applikation eines modernen Smartphones würde die Methode nicht angewendet werden
können und warum nicht?
\subpoints{3}

\begin{solution}{3}[blank]
Farbtabelle (Skript Teil 2, Folie 47)
\end{solution}

\item Ein \wordline[Lückentext]{8em} kann sehr einfach \wordline[korrigiert]{6em} werden und ist geeignet auch \wordline[komplexe]{4em} Sachverhalte abzufragen.

\end{enumerate}
