\documentclass{uulm-assignment}
\setboolean{showsolutions}{false} % Compile with or without answers

% \usepackage[utf-8]{inputenc}
\usepackage{subfigure}

\usepackage{amssymb}
\usepackage{hyperref}
\usepackage{tikz}

\faculty{Institut für Verteilte Systeme}
\course{Mobilkommunikation}
\semester{Wintersemester 2012/2013}
\supervisor{Stefan Dietzel,
	Dominik Lang,
	Thomas Lukaseder,
	Prof. Dr. Frank Kargl}
 
\assignmentdeadline{03.02.2013, 23:59}	 % Abgabedatum: XYZ
%\assignmentduration{15 Minuten} % Bearbeitungsdauer: XYZ
% \studentdata	              % Name & Matrikelnummer Feld

\assignmenttype{Übungsblatt}
\assignmentno{6}
\title{MANET-Simulation}

\begin{document}
\maketitle

\task{Aufgabe}{JiST/SWANS-Umgebung}{4}

Laden Sie die JiST/SWANS-Distribution aus ILIAS herunter. Übersetzen Sie die Sourcen mittels Ant: \verb|ant build|. Als eines der Beispielprotokolle enthält die Distribution eine Implementierung des AODV-Protokolls, welches Sie auch in der Vorlesung kennen gelernt haben.

\begin{solution}
Beispiel-Lösung für Übungsblatt. Wird nicht angezeigt, wenn \verb|showsolutions=false|
\end{solution}

\begin{solution}[5]
Beispiel-Lösung für Quiz. Wird mit 5 Linien angezeigt, wenn \verb|showsolutions=false|
\end{solution}

\begin{solution}[5][squared]
Beispiel-Lösung für Quiz. Wird mit 5 Reihen Karos angezeigt, wenn \verb|showsolutions=false|
\end{solution}

\begin{solution}[5][blank]
Beispiel-Lösung für Quiz. Wird mit 5 Reihen Leerraum angezeigt, wenn \verb|showsolutions=false|
\end{solution}

\end{document}
