\assignment{Datenbanksysteme}

\begin{enumerate}

\item 
Erläutern Sie den Begriff des Lost-Updates (verlorengegangene Änderung) im
Zusammenhang mit Transaktionen an einem Beispiel.

\begin{solution}{5}
Wenn zwei Transaktionen je den selben Wert eines Datums
lesen und unabhängig voneinander verändern und daraufhin wieder abspeichern,
geht die Aktualisierung der ersten speichernden Transaktion verloren. Beispiel:
T1 erhöht Wert um 1, T2 verdoppelt den Wert. Beide lesen Wert 4. T1 speichert
zuerst 5, T2 speichert danach 8 zurück. Die Aktualisierung auf 5 geht verloren.
\end{solution}

\subpoints{3}

\item 
Was versteht man unter dem Serialisierbarkeitsprinzip von Transaktionen? 

\begin{solution}{6}[false]
Eine nebenläufige Ausführung von Transaktionen ist genau dann korrekt, wenn es
eine serielle Ausführung dieser Transaktionen gibt, die vom selben
Ausgangszustand zum selben Endzustand kommt.
\end{solution}

\subpoints{5}

\item
Warum ist es sinnvoll beim Datenbankentwurf eine Normalform einzuhalten?

\begin{solution}{4}
Ein Entwurf gemäß Normalformen vermeidet Einfüge- und Löschanomalien, d.h. dass
unvollständige Daten nicht eingefügt oder wichtige Daten zusammen mit zu
löschenden Daten mitgelöscht werden müssen.
\end{solution}

\subpoints{2}

\end{enumerate}
